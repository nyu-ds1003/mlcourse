%% LyX 2.3.4.2 created this file.  For more info, see http://www.lyx.org/.
%% Do not edit unless you really know what you are doing.
\documentclass[english,dvipsnames,aspectratio=169]{beamer}
\usepackage{mathptmx}
\usepackage{eulervm}
\usepackage[T1]{fontenc}
\usepackage[latin9]{inputenc}
\usepackage{babel}
\usepackage{amstext}
\usepackage{amssymb}
\usepackage{graphicx}
\usepackage{ifthen}
\usepackage{xcolor}
\usepackage{xspace}
\usepackage{tikz}
\usetikzlibrary{tikzmark}
\usetikzlibrary{calc}
\usepackage{pgfplots}
%\pgfplotsset{compat=1.17}
\usepackage{booktabs}
\usepackage{xpatch}

\xpatchcmd{\itemize}
  {\def\makelabel}
  {\ifnum\@itemdepth=1\relax
     \setlength\itemsep{2ex}% separation for first level
   \else
     \ifnum\@itemdepth=2\relax
       \setlength\itemsep{1ex}% separation for second level
     \else
       \ifnum\@itemdepth=3\relax
         \setlength\itemsep{0.5ex}% separation for third level
   \fi\fi\fi\def\makelabel
  }
 {}
 {}

\ifx\hypersetup\undefined
  \AtBeginDocument{%
    \hypersetup{unicode=true,pdfusetitle,
 bookmarks=true,bookmarksnumbered=false,bookmarksopen=false,
 breaklinks=false,pdfborder={0 0 0},pdfborderstyle={},backref=false,colorlinks=true,
 allcolors=NYUPurple,urlcolor=LightPurple}
  }
\else
  \hypersetup{unicode=true,pdfusetitle,
 bookmarks=true,bookmarksnumbered=false,bookmarksopen=false,
 breaklinks=false,pdfborder={0 0 0},pdfborderstyle={},backref=false,colorlinks=true,
 allcolors=NYUPurple,urlcolor=LightPurple}
\fi

\makeatletter

%%%%%%%%%%%%%%%%%%%%%%%%%%%%%% LyX specific LaTeX commands.
%% Because html converters don't know tabularnewline
\providecommand{\tabularnewline}{\\}

%%%%%%%%%%%%%%%%%%%%%%%%%%%%%% Textclass specific LaTeX commands.
% this default might be overridden by plain title style
\newcommand\makebeamertitle{\frame{\maketitle}}%
% (ERT) argument for the TOC
\AtBeginDocument{%
  \let\origtableofcontents=\tableofcontents
  \def\tableofcontents{\@ifnextchar[{\origtableofcontents}{\gobbletableofcontents}}
  \def\gobbletableofcontents#1{\origtableofcontents}
}

%%%%%%%%%%%%%%%%%%%%%%%%%%%%%% User specified LaTeX commands.
\usetheme{CambridgeUS} 
\beamertemplatenavigationsymbolsempty


% Set Color ==============================
\definecolor{NYUPurple}{RGB}{87,6,140}
\definecolor{LightPurple}{RGB}{165,11,255}


\setbeamercolor{title}{fg=NYUPurple}
\setbeamercolor{frametitle}{fg=NYUPurple}

\setbeamercolor{background canvas}{fg=NYUPurple, bg=white}
\setbeamercolor{background}{fg=black, bg=NYUPurple}

\setbeamercolor{palette primary}{fg=black, bg=gray!30!white}
\setbeamercolor{palette secondary}{fg=black, bg=gray!20!white}
\setbeamercolor{palette tertiary}{fg=gray!20!white, bg=NYUPurple}

\setbeamertemplate{headline}{}
\setbeamerfont{itemize/enumerate body}{}
\setbeamerfont{itemize/enumerate subbody}{size=\normalsize}

\setbeamercolor{parttitle}{fg=NYUPurple}
\setbeamercolor{sectiontitle}{fg=NYUPurple}
\setbeamercolor{sectionname}{fg=NYUPurple}
\setbeamercolor{section page}{fg=NYUPurple}
%\setbeamercolor{description item}{fg=NYUPurple}
%\setbeamercolor{block title}{fg=NYUPurple}

\setbeamertemplate{blocks}[rounded][shadow=false]
\setbeamercolor{block body}{bg=normal text.bg!90!NYUPurple}
\setbeamercolor{block title}{bg=NYUPurple!30, fg=NYUPurple}



\AtBeginSection[]{
  \begin{frame}
  \vfill
  \centering
\setbeamercolor{section title}{fg=NYUPurple}
 \begin{beamercolorbox}[sep=8pt,center,shadow=true,rounded=true]{title}
    \usebeamerfont{title}\usebeamercolor[fg]{title}\insertsectionhead\par%
  \end{beamercolorbox}
  \vfill
  \end{frame}
}

\makeatother

\setlength{\parskip}{\medskipamount} 

\input ../macros

\begin{document}
\input ../rosenberg-macros

\title[DS-GA 1003]{Course Overview}
\author{Ravid Shwartz-Ziv}
\date{Jan 24, 2023}
\institute{CDS, NYU}

\makebeamertitle
\mode<article>{Just in article version}

\begin{frame}{Contents}
\tableofcontents{}
\end{frame}

\section{Logistics}
\begin{frame}{Course Staff}
\begin{itemize}
\item Instructors:
    \begin{itemize}
        \item Ravid Shwartz Ziv
        \item Mengye Ren
    \end{itemize}

\item Section leaders:
    \begin{itemize}
        \item Colin Wan
        \item Ying Wang
        \item  Yanlai Yang
    \end{itemize}

\item Graders:
    \begin{itemize}
        \item Xiaojing Fan
        \item Junze Li
        \item Richard-John Lin
        \item Ying Wang
        \item Jerry Xue
        \item Frances Yuan
    \end{itemize}
\end{itemize}
\end{frame}

\begin{frame}{Logistics}
\begin{itemize}
\item Class webpage: \url{https://nyu-ds1003.github.io/spring2023} 
\begin{itemize}
    \item Course materials (lecture slides, homework assignments) will be made available on the website
\end{itemize}
\item Announcements via Brightspace
\item Discussion / questions on CampusWire: \url{https://campuswire.com/c/G0F20206F/feed}
\item Sign up to Gradescope to submit homework assignments (entry code \textbf{475536})
\end{itemize}

    \begin{itemize}
\item Office Hours: By appointment 
\begin{itemize}
    \item  Ravid: Tuesday 9:00-10:00 am; Mengye: Tuesday 2:00-3:00 pm
\item  Colin: Monday 1:00-2:00 pm; Ying :Wednsday 6:00 pm - 7:00 pm;  Yanlai : Wednsday 1:00pm-2:00 pm; Room 204, 60 5th Ave; 
\end{itemize}
    \end{itemize}
\end{frame}

\begin{frame}{Assessment}
\begin{itemize}
\item 7-8 assignments ($40\%$)
\item Two tests ($60\%$)
\begin{itemize}
\item Midterm Exam ($30\%$)
\item Final Exam ($30\%$)
\end{itemize}

\item Extra credits ($2\%$) answer other students' questions in a substantial and helpful way on Campuswire

\end{itemize}
\end{frame}
%
\begin{frame}{Homework}
\begin{itemize}
    \item Submit through Gradescope as a \textbf{PDF document}
    \item Late policy: You have seven late days in total which can be used throughout the semester without penalty (see more details on website).

\item You can collaborate with other students on the homework assignments, but please:
\begin{itemize}
\item Write up the solutions and code on your own;
\item And list the names of the students you discussed each problem with.
\end{itemize}

\end{itemize}
\end{frame}


\begin{frame}{Prerequisites}
\begin{itemize}
\item DS-GA 1001: Introduction to Data Science 
\item DS-GA 1002: Statistical and Mathematical Methods
\item Math
\begin{itemize}
\item Multivariate Calculus
\item Linear Algebra
\item Probability Theory
\item Statistics
\item {[}Preferred{]} Proof-based linear algebra or real analysis
\end{itemize}
\item Python programming (numpy)
\end{itemize}
\end{frame}


\section{Course Overview and Goals}
\begin{frame}{Syllabus (Tentative)}

13 weeks of instruction + 1 week midterm exam
\begin{itemize}
    \item 2 weeks: introduction to \textbf{statistical learning theory}, \textbf{optimization}

\item 2--3 weeks: \textbf{Linear }methods for binary classification
and regression (also\textbf{ kernel methods)}

\item 2 weeks: \textbf{Probabilistic models}, \textbf{Bayesian}
methods

\item 1 week: \textbf{Multiclass} classification and introduction to \textbf{structured
prediction}

\item 3--4 weeks: \textbf{Nonlinear} methods (\textbf{trees}, \textbf{ensemble}
methods, and \textbf{neural networks})

\item 2 weeks: \textbf{Unsupervised} learning: \textbf{clustering} and \textbf{latent variable} models

\item More detailed schedule on the course website (still subject to change)

\item Certain applications and practical algorithms may be covered in the labs
\end{itemize}
\end{frame}
%
\begin{frame}{The high level goals of the class}
\begin{itemize}
\item Our focus will be on the fundamental building blocks of machine learning
\item ML methods have a lot of names; our goal is for you to notice that\\
\begin{itemize}
\item[] \textbf{fancy new method A ``is just'' familiar thing B + familiar
thing C + tweak D}
\end{itemize}
        \begin{itemize}
\item SVM ``\textbf{is just}'' ERM with hinge loss with $\ell_{2}$ regularization
\item Pegasos ``\textbf{is just}'' SVM with SGD with a particular step
size rule
\item Random forests ``\textbf{are just}'' bagging with trees, with a different
approach to choosing splitting variables
        \end{itemize}
\end{itemize}
\end{frame}
%
\begin{frame}{The level of the class}
\begin{itemize}
\item We will learn how to implement each ML algorithm \textbf{from scratch} using numpy alone, without any ML libraries.
\item Once we have implemented an algorithm from scratch once, we will use the sklearn version.
\end{itemize}
\end{frame}
\end{document}
