%% LyX 2.3.6.2 created this file.  For more info, see http://www.lyx.org/.
%% Do not edit unless you really know what you are doing.
\documentclass[english,aspectratio=169]{beamer}
\usepackage{mathptmx}
\usepackage{eulervm}
\usepackage[T1]{fontenc}
\usepackage[latin9]{inputenc}
\usepackage{babel}
\usepackage{amstext}
\usepackage{amssymb}
\usepackage{graphicx}
\ifx\hypersetup\undefined
  \AtBeginDocument{%
    \hypersetup{unicode=true,pdfusetitle,
 bookmarks=true,bookmarksnumbered=false,bookmarksopen=false,
 breaklinks=false,pdfborder={0 0 0},pdfborderstyle={},backref=false,colorlinks=true,
 allcolors=NYUPurple,urlcolor=LightPurple}
  }
\else
  \hypersetup{unicode=true,pdfusetitle,
 bookmarks=true,bookmarksnumbered=false,bookmarksopen=false,
 breaklinks=false,pdfborder={0 0 0},pdfborderstyle={},backref=false,colorlinks=true,
 allcolors=NYUPurple,urlcolor=LightPurple}
\fi

\makeatletter


%%%%%%%%%%%%%%%%%%%%%%%%%%%%%% LyX specific LaTeX commands.
%% Because html converters don't know tabularnewline
\providecommand{\tabularnewline}{\\}

%%%%%%%%%%%%%%%%%%%%%%%%%%%%%% Textclass specific LaTeX commands.
% this default might be overridden by plain title style
\newcommand\makebeamertitle{\frame{\maketitle}}%
% (ERT) argument for the TOC
\AtBeginDocument{%
  \let\origtableofcontents=\tableofcontents
  \def\tableofcontents{\@ifnextchar[{\origtableofcontents}{\gobbletableofcontents}}
  \def\gobbletableofcontents#1{\origtableofcontents}
}

%%%%%%%%%%%%%%%%%%%%%%%%%%%%%% User specified LaTeX commands.
\usetheme{CambridgeUS} 
\beamertemplatenavigationsymbolsempty


% Set Color ==============================
\definecolor{NYUPurple}{RGB}{87,6,140}
\definecolor{LightPurple}{RGB}{165,11,255}


\setbeamercolor{title}{fg=NYUPurple}
%\setbeamercolor{frametitle}{fg=NYUPurple}
\setbeamercolor{frametitle}{fg=NYUPurple}

\setbeamercolor{background canvas}{fg=NYUPurple, bg=white}
\setbeamercolor{background}{fg=black, bg=NYUPurple}

\setbeamercolor{palette primary}{fg=black, bg=gray!30!white}
\setbeamercolor{palette secondary}{fg=black, bg=gray!20!white}
\setbeamercolor{palette tertiary}{fg=gray!20!white, bg=NYUPurple}

\setbeamertemplate{headline}{}

\setbeamercolor{parttitle}{fg=NYUPurple}
\setbeamercolor{sectiontitle}{fg=NYUPurple}
\setbeamercolor{sectionname}{fg=NYUPurple}
\setbeamercolor{section page}{fg=NYUPurple}

\AtBeginSection[]{
  \begin{frame}
    \frametitle{Table of Contents}
    \tableofcontents[currentsection]
  \end{frame}

  % \begin{frame}
  % \vfill
  % \centering
  % \begin{beamercolorbox}[sep=8pt,center,shadow=true,rounded=true]{title}
  %   \usebeamerfont{title}\insertsectionhead\par%
  % \end{beamercolorbox}
  % \vfill
  % \end{frame}
}

\makeatother

\begin{document}
\input{../rosenberg-macros.tex}
\title[DS-GA 1003 ]{Probabilistic models\\
- \\
Bayesian Methods \\
- \\
Discussion}
\author{Marylou Gabri\'e}
\date{March 16, 2021}
\institute{CDS, NYU}

% \title[DS-GA 1003]{Bayesian Methods}
% \author{Julia Kempe \& David S. Rosenberg }
% \date{March 26, 2019}
% \institute{CDS, NYU}

\makebeamertitle
\mode<article>{Just in article version}
\begin{frame}{Bayesian decision for absolute loss is median}
   
 \end{frame}
% 
\begin{frame}{Hours of sun during January in Vienna}
\begin{itemize}
    \item $N=30$ measurements $y_i$
    \item What could you choose as a parametric family?
    \item What is the MLE?
\end{itemize}

\begin{center}
    \includegraphics[height=0.75\textheight]{figs/histogram.pdf}
    \par\end{center}
\end{frame}
%
\begin{frame}{Hours of sun during January in Vienna}
    \begin{itemize}
        \item $N=30$ measurements $y_i$
        \item What could you choose as a parametric family?
        \item What is the MLE?
    \end{itemize}
    
    \begin{center}
        \includegraphics[height=0.75\textheight]{figs/histogram_density.pdf}
        \par\end{center}
\end{frame}
%
\begin{frame}{Hours of sun during January in Vienna}
        \begin{itemize}
            \item $N=30$ measurements $y_i$
            \item Now meteorologists tells you they have a prior: $p(\mu) = \mathcal{N}(\mu; \mu_0,1)$
            \item What is the posterior?
            \item[] 
            \item[] 
            \item[] 
            \item[] 
            \item[] 
            \item[] 
            \item[] 
            \item[] 
            \item[] 
            \item[] 
        \end{itemize}
        
%         \begin{center}
%             \includegraphics[height=0.75\textheight]{figs/histogram.pdf}
%             \par\end{center}
\end{frame}
%
\begin{frame}{Hours of sun during January in Vienna}
    \begin{itemize}
            \item $N=30$ measurements $y_i$
            \item Now meteorologists tells you they have a prior: $p(\mu) = \mathcal{N}(\mu; \mu_0,1)$
            \item What is the posterior?
    \end{itemize}
    \begin{center}
        \includegraphics[height=0.55\textheight]{figs/histogram_density.pdf}
        \includegraphics[height=0.55\textheight]{figs/histogram_posterior.pdf}
    \par\end{center}
\end{frame}

\begin{frame}{Hours of sun during January in Vienna}
    \begin{itemize}
            \item $N=30$ measurements $y_i$
            \item Now meteorologists tells you they have a prior: $p(\mu) = \mathcal{N}(\mu; \mu_0,1)$
            \item What is the posterior? What is the credible set?
    \end{itemize}
    \begin{center}
        \includegraphics[height=0.55\textheight]{figs/histogram_density.pdf}
        \includegraphics[height=0.55\textheight]{figs/histogram_posterior_zoom.pdf}
    \par\end{center}
\end{frame}

\begin{frame}{Hours of sun during January in Vienna}
    \begin{itemize}
        \item $N=30$ measurements $y_i$
        \item Now meteorologists tells you they have a prior: $p(\mu) = \mathcal{N}(\mu; \mu_0,1)$
        \item The posterior is $p(\mu|\{y_i\}_{i=1}^N, \sigma^2) = \mathcal{N}(\mu; \mu', \sigma'^2)$.
        \item What are the point estimates of $\mu$ minimizing squared loss, absolute loss and 0-1 loss?
        \item[] 
        \item[] 
        \item[] 
        \item[] 
        \item[] 
        \item[] 
        \item[] 
        \item[] 
        \item[] 
    \end{itemize}
\end{frame}

\end{document}

